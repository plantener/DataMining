\section{Discussion}
We have described the attributes both by using histograms and box-plots. The box plots leaves us with some concern to whether or not all of the attributes are actually right. For some of our attributes, there are several outliers, and some of them could indicate problems with some of the data. However we have still been able to work with the data.

Furthermore we have plotted some of the attributes against each other in order to see how they interact with each other, and how they effect the probability of having the disease. We have also looked at how the attributes correlate with the fact whether a person has the disease. We see that age is the attribute having the most effect at whether a person has the disease or not. We see that none of the attributes have a high correlation, however most of them have some correlation.

Likewise looking at the principal components, we could see that people having high values in the attributes were more likely to have the disease.

To sum up, it is hard to give a specific formula for how the disease depend at the attributes. However a pattern emerges that the more adiposity and the higher age, and so on for the other attributes, the more likely you are to get the disease. We do not see a direct dependency for the attributes, but it seems like people can lower the probability of getting the disease by living healthy.