\section{Correlation of the data}

Below in Figure \ref{correlationTable}, we see how the attributes are related to each other. For instance, we can see that obesity is highly related to adiposity, however the most interesting thing is to see how the attributes might have an impact on whether the subject has the disease or not. So the correlations to chd are the most interesting to look at.

\begin{figure}[H]
\begin{longtable}{ l c c c c c c c c c c}
  \hline                        
  \emph{Correlation} &  \emph{sbp} &  \emph{tobacco} &  \emph{ldl} &  \emph{adiposity} &  \emph{famhist} &  \emph{typea} &  \emph{obesity} &  \emph{alcohol} &  \emph{age} &  \emph{chd}	\\ \hline
sbp & 1 & 0.2122 & 0.1583 & 0.3565 &  0.0856 & -0.0575 &
   0.2381 & 0.1401 &  0.3888 & 0.1924 \\ 
   
 tobacco & 0.2122 & 1 & 0.1589 &  0.2866 &  0.0886 & -0.0146 &
   0.1245 &  0.2008 & 0.4503 &  0.2997 \\ 
   
ldl & 0.1583 & 0.1589 &  1 &          0.4404 &  0.1613 &  0.0440 &
   0.3305 &  -0.0334 &   0.3118 & 0.2630 \\ 

adiposity &  0.3565 & 0.2866 & 0.4404 & 1 &          0.1817 & -0.0431
 &  0.7166 &  0.1003 &  0.6260 & 0.2541 \\ 

famhist & 0.0856 & 0.0886 &  0.1614 &  0.1817 &  1      &    0.0448 &
   0.1156 & 0.0805 &  0.2397 &  0.2724 \\ 

typea & -0.0575 & -0.0146 &  0.0440 & -0.0431 &  0.0448 &  1 &
   0.0740 &   0.0395 & -0.1026 &  0.1032 \\ 

obesity & 0.2381 &  0.1245 &  0.3305 &  0.7166 &  0.1156 &  0.0740 &
   1 &          0.0516 &  0.2918 &  0.1001 \\ 

alcohol & 0.1401 & 0.2008 & -0.0334 &   0.1003 &  0.0805 &  0.0395 &
   0.0516 &  1 &          0.1011 &  0.0625 \\ 

age & 0.3888 &  0.4503 &  0.3118 &  0.6260 & 0.2397 & -0.1026 &
   0.2918 &  0.1011 &  1     &     0.3730  \\ 
  
chd &  0.1924 & 0.2997 & 0.2631 & 0.2541 & 0.2723 & 0.1032 & 0.1001 & 0.0625 & 0.3730 & 1 \\ \hline
\end{longtable}
\caption{Table showing how the attributes are correlated to whether subjects have positive or negative chd.}
\label{correlationTable}
\end{figure}

We see that the CHD attribute is not totally dependent on any of the other attributes, but the attribute with the most impact is the age. On the other hand we see that alcohol has the smallest correlation to CHD. So this indicates that age has greater influence compared to alcohol, when to get the possibility of a man having the disease. In fact this table shows that the alcohol consumption has very small meaning for whether a person has the disease.