\section{Description}
There is a high level of South Africans with the heart disease \textit{Coronary Heart Disease (CHD)}\footnote{http://www-stat.stanford.edu/~tibs/ElemStatLearn/datasets/SAheart.info.txt}. The idea behind the dataset is to see if there is any consistency to why this is a high-risk CHD zone, and to measure what effects the treatment they are given have. The dataset used is a sample of a larger dataset, and consists of 462 rows, containing various information from different persons.

The dataset was described in the South African Medical Journal in 1983, and performed by: \textit{Rossouw JE, Du Plessis JP, Benadé AJ, Jordaan PC, Kotzé JP, Jooste PL, Ferreira JJ}\footnote{http://europepmc.org/abstract/MED/6623218/reload=0;jsessionid=MT3wbNbXg1PKSt1adMMW.10}

We have not been able to recover the journal, as it does not seem to be published on the internet, which means that we do not know what exactly has been done to the data, and what the results of their analysis was.

The sample of the dataset has been obtained from: \textit{http://www-stat.stanford.edu/~tibs/ElemStatLearn/} and is known as: \textit{South African Heart Disease}

\paragraph{We envision that,} after having analyzed the data, we will be able to do a classification of a person, to whether or not he will have CHD. Let us say, we know the systolic blood pressure, tobacco consumption, low density lipoprotein cholesterol, adiposity, history of heart disease in the family, type-a score, obesity, alcohol consumption and age of a given person, then it could be interesting to classify whether or not he is likely to have the disease.%, based on the Principal Component Analysis and correlation between attributes.

Using regression analysis, we can predict some of the attributes, based on the others. So knowing the values of the other attributes, one could be able to calculate the value of another attribute.% I.e given that we know a persons cholesterol number, and his age, we can predict the systolic blood pressure.

Using clustering analysis, we can group persons into categories based on their similarities. This can be done using patterns from our dataset. This could be interesting to see if we have a clustering of subjects, where most of them are sick, or most of them are healthy. If we plot some of the data, it could be interesting if we get for instance a cluster of ill people.

Using association mining, we can discover patterns. For instance, one can check how the presence of heart disease in the family relate to whether the person itself has the disease.% like, if a person is older than 40, and has i high cholesterol number, he might be likely to drink alcohol.

Using anomaly detection, it could be possible to identify outliers in the data. By this we can see if we abnormal data and outliers, the data might be erroneous. Furthermore this might lead to discovering of whether the person is generally healthy. Eg. if he has very high cholesterol or very high tobacco consumption, he might not necessarily have chd, but he might have other health problems.% Based on the pattern from the data, it is possible to see if a reading from the person is out-of-normal, and do something about it.

%Finally the most interesting thing is to be able to classify whether a person has the disease or not based on his systolic blood pressure, tobacco consumption, low density lipoprotein cholesterol, adiposity, history of heart disease in his family, type-a score, obesity, alcohol consumption and age. This is what we have data of for our subjects, and based on the data set one could see if there is any dependency of these and whether a person has the coronary heart disease. Therefore this data set should give an indication of how one can classify the probability of a person having the disease based these attributes.