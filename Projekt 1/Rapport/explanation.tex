\section{Explanation}
In this section we will explain all of the different attributes, our date set contains, as well as how we came to the different types and whether they are discrete or continues attributes. We will also discuss the subject of data issues i.e. missing and corrupted data.

\subsection{Attributes}

\paragraph{Row} is describing the id of the subject. This attribute is \textbf{discrete} and \textbf{nominal}.

\paragraph{SBP} is Systolic blood pressure which is a measurement of the subjects blood pressure, and has the unit mmHg. The desired SBP lies in the range 90-119, but it is plausible to go below or above, but that is when there usually is heart problems.

We came to the type \textbf{interval} because zero as blood pressure is not the lack of pressure, but instead equals 1 atmosphere. Furthermore even though systolic blood pressure normally is described as a real number and therefore as a continuous attribute, for our case we only have natural numbers, meaning the attribute is \textbf{discrete}.%Since we came to interval then the attribute most likely would be \textbf{continues}.

\paragraph{Tobacco} is measured as the cumulative consumption in kg, and since the starting point weight is at 0 kg, and 2 kg is double 1 kg, it would make the attribute type \textbf{ratio}.  The attribute is described as real numbers, it is a \textbf{continuous} attribute.

\paragraph{LDL} stands for Low density lipoprotein cholesterol and is measured in mmol/L. The regular LDL numbers lies around 4.9 mmol/L. It is a \textbf{continuous} attribute, with a \textbf{ratio} type since zero is the absence of LDL and is described by real numbers.

\paragraph{Adiposity} is a sort of indexing for a persons body mass, where this is taken from the waist. It is a \textbf{continuous} attribute since we have real numbers, and it takes on \textbf{ratio} as type. For a man, this value should range from 8\% - 25\% to be in a healthy state, and for a woman it should be 21\% -38\%.

\paragraph{Famhist} short for family history is the attribute that tells if heart disease is present or absent in the persons family, this also make attribute \textbf{discrete} because it is represented as a set of words in the dataset. The type is \textbf{nominal} because it can either be there or not there. We cannot rank the values, we can only put them into categories.

\paragraph{Type-a} is a test score that shows how aggressive a person is. This is a \textbf{discrete} attribute since it is a countable value from 0-100. Because a test score usually is a form of grading system, it would make the type \textbf{ordinal}.

\paragraph{Obesity} Obesity is like adiposity an index for a persons body mass, we better know this one as BMI. It is as adiposity a \textbf{continuous} attribute with the \textbf{ratio} type, since 0 means absence of body mass. This should lie between 18.5 -25 to be within the healthy region of the scale.

\paragraph{Alcohol} is measured as the current alcohol consumption, which we believe is measured as pure alcohol litres consumption over a year. Since we have real numbers and infinitely many possible values, it is a \textbf{continuous} attribute, and because 0 litres of pure alcohol is a absence of what is measured, the type is \textbf{ratio}.

\paragraph{Age} is the age of the subject. It is usually handled as a continues attribute, but in this data set it is treated like a \textbf{discrete} attribute. We have the age in years, meaning we only have the natural numbers. Age has a starting point at 0 which would make it absence of a age and that would make the attribute type \textbf{ratio}.

\paragraph{CHD} stands for coronary heart disease and is a binary attribute that says if the subject has CHD. Since it is binary it makes the attribute \textbf{discrete} and the type would be \textbf{nominal}.

\subsection{Data issues}
As of missing data, we don't really have any corrupted or missing data in any of our data entries, but we do have a single missing data entry at 262. Since it is an entire entry that is missing it will not do any harm to our data models.

Outliers will be covered in section \ref{boxplotSection} concerning boxplots of the attributes.
