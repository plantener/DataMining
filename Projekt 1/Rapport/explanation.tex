\section{Explanation}

In this section we will explain all of the different attributes, our date set contains, as well as how we came to the different types and whether they are discrete or continues attributes. We will also get to subject of data issues e.i. missing and corrupted data.

\subsection{Attributes}

\paragraph{Row} This field is describing the id of the subject. This attribute is discrete and nominal.

\paragraph{SBP} SBP is Systolic blood pressure which is a measurement of the subjects blood pressure, and has the unit mmHg. The desired SBP lies in the range 90-119, but it is plausible to go below or above, but that is when there usually is heart problems.
We came to the type \textbf{interval} because it doesn't originate at zero. Since we came to interval then the attribute most likely would be \textbf{continues}.

\paragraph{Tobacco} Tobacco is measured as the cumulative consumption in kg, and since that the starting point weight is at 0 it would make the attribute type \textbf{ratio}. Since it is represented as a float it is a \textbf{continues} attribute.

\paragraph{LDL} LDL stands for Low density lipoprotein cholesterol and is measured in mmol/L. The regular LDL numbers lies around 4.9 mmol/L. It is a \textbf{continuous} attribute, with a \textbf{ratio} type since zero is the absence of LDL and is  represented as a float.

\paragraph{Adiposity} Adiposity is a sort of indexing for a persons body mass, where this is taken from the waist. It is a \textbf{continuous} attribute since it has floating point attributes, and it takes on \textbf{ratio} as type.

\paragraph{Famhist} Family history is the attribute that tells if heart disease is present or absent in the persons family, this also make attribute \textbf{discrete} because it is represented as a set of words in the dataset. The type is \textbf{nominal} because it can either be there or not there.

\paragraph{Type-a} Type-a is test score which make it a \textbf{discrete} attribute since it is a countable value from 0-100. Because a test score usually is a form of grading system, which would make the type \textbf{ordinal}.

\paragraph{Obesity} Obesity is like adiposity a indexing for a persons body mass, we better know this one as BMI. Where adiposity uses the waist to find the index, obesity uses the height. It is as adiposity a \textbf{continuous} attribute with the \textbf{ratio} type.

\paragraph{Alcohol} Alcohol is measured as the current alcohol consumption, which we believe is measured as pure alcohol litres consumption over a year. Since it is represented as a float it is most likely a \textbf{continuous} attribute, and because 0 l of pure alcohol is a absence of what is measured which would make the type \textbf{ratio}.

\paragraph{Age} Age is the age of the subject. It is usually handled as a continues attribute, but in this data set it is treated like a \textbf{discrete} attribute. Age has a starting point at 0 which would make it absence of a age and that would make the attribute type \textbf{ratio}.

\paragraph{CHD} CHD stands for coronary heart disease and is a binary attribute that says if the subject has CHD. Since it is binary it makes the attribute \textbf{discrete} and the type would be \textbf{nominal}.

\subsection{Data issues}

As of missing data, we don't really have any corrupted or missing data in any of our data entries, but we do have a single missing data entry at 262. Since it is a entire entry that is missing it wont do any harm to our data models. We will