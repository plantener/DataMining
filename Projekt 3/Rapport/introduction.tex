\section{Introduction}
The following report will try to describe how well the data can be clustered using Gaussian Mixture Model and hierarchical clustering. Using these clusters we will try to extract information on whether a person has CHD. We will try to see if a subject located in a specific cluster, will be more or less likely to have the disease.

Using Association Mining on our data, we will extract association rules using the apriori algorithm, to say something about when an object is likely to have CHD. We will also look for other strong associations that could explain some of the data.

Finally we will discuss outlier detection using Gaussion Kernel Density, K-Nearest Neighbour density, K-Nearest Neughbour average relative density and distance to Kth(5th??) Nearest Neighbour. Based on these measures we will evaluate if our dataset contains outliers. Furthermore we can see which objects each method is likely to detect as outlier, and such we will see if certain objects recurs for the different methods.

Here is a little recap of the attributes used in the dataset:
\begin{itemize}
\item Age - Persons age
\item SBP - Systolic Blood Pressure
\item Alcohol - Alcohol consumption
\item Tobacco - Tobacco usage
\item LDL - Low density lipoprotein cholesterol
\item Adiposity - Index of body fat
\item Obesity - Index of body mass
\item Famhist - Family History. Has the famility a history of CHD
\item Type-A - Aggresiveness index
\item CHD - Coronary Heart Disease
\end{itemize}
