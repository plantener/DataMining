\documentclass[11pt,a4paper,english]{article}
%\usepackage[ansinew]{inputenc}
\usepackage[T1]{fontenc}
%\usepackage[english]{babel}
\usepackage[utf8]{inputenc}
\usepackage[danish]{babel}
\usepackage{amsmath,dcolumn,booktabs} %%,cellspace}
\usepackage{amssymb}
\usepackage{amsthm}
\newtheorem{mydef}{Definition}
\usepackage{marvosym}
\usepackage{tocloft}
\usepackage{a4wide}
\usepackage{graphicx}
\usepackage{float}
%%\usepackage{threeparttable}
\usepackage[tableposition=top]{caption}
\usepackage{enumerate}
\usepackage{clrscode}
\usepackage{longtable}
\usepackage[table]{xcolor}
\usepackage{lastpage}
\usepackage{caption}
\usepackage{mathtools}
%%\usepackage{stmaryrd}
%\usepackage{ulem}
\usepackage{fancyhdr}
\pagestyle{fancy}
\usepackage{array}
\usepackage{listings}
\usepackage{color}
\usepackage{colortbl}
\usepackage{multirow}
%%\usepackage{blkarray}
\usepackage{algorithmic}
\usepackage{algorithm}
\usepackage{hhline,array}
\usepackage{footnote}
%\usepackage{pdfpages}
\usepackage[official]{eurosym}
\usepackage{rotating}
\usepackage{boxedminipage}
\addtolength\tabcolsep{-3pt}%
\usepackage{cellspace,amsmath}
\addtolength\cellspacetoplimit{2pt}
\addtolength\cellspacebottomlimit{2pt}
\usepackage{subcaption}
\usepackage{array}

\definecolor{myColor}{RGB}{232,238,246}
\definecolor{comcol}{rgb}{0.1, 0.5, 0.1}
\definecolor{stringcol}{rgb}{0.75, 0.1, 0.95}

\newenvironment{changemargin}[2]{%
\begin{list}{}{%
\setlength{\topsep}{0pt}%
\setlength{\leftmargin}{#1}%
\setlength{\rightmargin}{#2}%
\setlength{\listparindent}{\parindent}%
\setlength{\itemindent}{\parindent}%
\setlength{\parsep}{\parskip}%
}%
\item[]}{\end{list}}

\lstset{language=java, keywordstyle=\color{blue}, commentstyle=\color{comcol}, stringstyle=\color{stringcol}, breaklines=true, showstringspaces=false}

%----------Define color: ------------
\definecolor{Gray}{rgb}{0.5,0.5,0.5}

%\usepackage{program}
%\numberwithin{algorithm}{chapter} hvis algoritmer skal gives navn efter kapitel
\usepackage[official]{eurosym}
%%\addtolength\cellspacetoplimit{4pt}
%%\addtolength\cellspacebottomlimit{4pt}
%%\usepackage{cite}
%\bibliographystyle{plane}
\usepackage{natbib}
\usepackage[pdftex]{hyperref}
\hypersetup{ %bookmarks in latex
    colorlinks,
    citecolor=black,
    filecolor=black,
    linkcolor=black,
    urlcolor=black
}

\newcommand{\insertfigure}[4]
{
    \begin{figure}[!hbt]
    \begin{center}
    \includegraphics[scale=#3]{#1}
    \end{center}
    \caption{#2}
    \label{#4}
    \end{figure}
}
%til listing pakken
\lstset{ %
%language=,                % choose the language of the code
basicstyle=\footnotesize,       % the size of the fonts that are used for the code
numbers=none,                   % where to put the line-numbers
numberstyle=\footnotesize,      % the size of the fonts that are used for the line-numbers
stepnumber=2,                   % the step between two line-numbers. If it's 1 each line will be numbered
numbersep=5pt,                  % how far the line-numbers are from the code
backgroundcolor=\color{white},  % choose the background color. You must add \usepackage{color}
showspaces=false,               % show spaces adding particular underscores
showstringspaces=false,         % underline spaces within strings
showtabs=false,                 % show tabs within strings adding particular underscores
frame=false,	                % adds a frame around the code
tabsize=2,	                % sets default tabsize to 2 spaces
captionpos=b,                   % sets the caption-position to bottom
breaklines=true,                % sets automatic line breaking
breakatwhitespace=false,        % sets if automatic breaks should only happen at whitespace
escapeinside={\%*}{*)}          % if you want to add a comment within your code
}


\newfloat{mathmodel}{htbp}{lom}
\floatname{mathmodel}{Model}

\makeatletter \let\c@mathmodel\c@equation
\newenvironment{model}[1]{%
	\begin{mathmodel}%
		\caption{#1}%
		\protected@edef\theparentequation{\theequation}%
  		\setcounter{parentequation}{\value{equation}}%
  		\setcounter{equation}{0}%
  		\def\theequation{\theparentequation\alph{equation}}%
  		\ignorespaces%
}%
{%
		\setcounter{equation}{\value{parentequation}}%
	\end{mathmodel}%
  	\ignorespacesafterend
}

\makeatother

\newcommand{\Max}{\operatorname{Max}}
\newcommand{\Min}{\operatorname{Min}}
\newcommand{\St}{\operatorname{S.t.}}
