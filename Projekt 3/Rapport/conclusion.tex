\section{Conclusion}

We have performed clustering for our data set. We do not see that our data set is very applicable for clustering, as we often get one very big cluster, and a few small ones. This is a result of our the majority of our data points are located closely to each other. However we see that we can get some smaller clusters where the majority of objects are CHD-positive. This is interesting as generally for our data set, the majority are CHD-negative.

Furthermore we have performed association mining. For this, we see that it is not easy for our data to say anything certain about having high values in some attributes leads to being CHD-negative or CHD-positive. We saw that having a high age, high consumption of tobacco and/or high LDL might increase the chance of being CHD-positive. However the association rules for this have a low support, so it is hard to say anything with certainty.

Lastly we have detected for outliers in our data set. We have used different approaches to find outliers, and we have seen that many of these approaches suggest some of the same subjects to be outliers. This could indicate that some of these persons in fact are outliers. However this is not necessarily true, as it might just be persons with unusual attribute-values. If a person is sick, he would probably also be more likely to have unhealthy values, which might lead him to be detected as an outlier.